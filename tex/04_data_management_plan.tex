\documentclass[12pt, class=article, crop=false]{standalone}
\usepackage[subpreambles=true]{standalone}
\usepackage[T1]{fontenc} % for font setting
\usepackage{newtxtext,newtxmath}
\usepackage{import,
            graphicx,
            parskip,
            url,
            amsmath,
            wrapfig,
            fancyhdr,
            xcolor}
\usepackage[most]{tcolorbox}
\usepackage[top=2.54cm, bottom=2.54cm, left=2.54cm, right=2.54cm, marginparwidth=2cm]{geometry}%set margin
\usepackage[sort&compress]{natbib}
\setcitestyle{square}
\setcitestyle{comma}
\usepackage[font={sf, small}, labelfont={bf, small}]{caption}

\tcbuselibrary{breakable}
\bibliographystyle{bibstyle}
\graphicspath{{images/}}

% header
\pagestyle{fancy}
\fancyhead[L]{Data Management Plan}
\fancyhead[R]{}

\begin{document}
\section*{Data Management Plan}

\subsubsection*{Data/Code Management}

\textbf{Data}: While we do not collect data in the field, we use field observations of stream chemistry and fish community compositions available in public data repositories. Data and metadata content and format will adhere to established standards, ensuring that the data are Findable, Accessible, Interoperable, and Reproducible (FAIR), as applied to the field of hydrology by the HydroShare Data Repository.

Throughout the project, data will exist in three forms.
\textit{Raw data}: This data is acquired immediately after collection from data repositories, prior to quality control and assurance checks.
\textit{Intermediate data}: Raw data that has undergone quality assurance checks.
It lists all raw data and flags data that did not pass quality checks, providing justification for exclusion.
\textit{Final data}: This is a subset of raw data considered of high enough quality for analysis.

The original data sourced from data repositories will be saved as spreadsheet (.csv) or R files (.rds) with ``raw'' in the filename.
Metadata from corresponding data repositories or through communications with data providers will be saved as a "README" markdown file (.md) in the same repository, allowing users to seamlessly access that information as needed.
All data will be hosted in Github along with source codes for analysis so readers can reproduce research findings.
We will track any changes to data using Github's version control system.

\textbf{Source Code}: We will develop and store our source codes for data formatting and statistical analysis in either R or Python scripts.
These scripts will be hosted on GitHub, ensuring version control and collaboration among project team members. 
GitHub's version control system will enable us to track changes, collaborate seamlessly, and maintain a historical record of code modifications throughout the project's duration.
To ensure that our research is accessible to the wider community, we commit to sharing our source codes as open-source. These codes will be made available under permissive licenses such as MIT or BSD. This allows others to use, modify, and build upon our work.

\textbf{Geospatial Data}: Our project heavily relies on Geospatial Information Systems (GIS). Geospatial data, which may include formats such as shapefiles, RDS (R Data Storage), or GeoTIFF, will be managed locally during the project.
The original data sourced from data repositories will be saved with ``raw'' in the filename.

\textcolor{red}{\textbf{Data Storage}:
Our project will generate a substantial amount of digital data, and we will use a two-phase strategy to minimize the risk of data loss.
During the data analysis phase (the temporal phase), we will store data locally.
These temporary files will be regularly backed up to cloud storage (e.g., OneDrive) and external hard drives by each PD.
Upon completion of the analysis (the permanent phase), all files (including source and intermediate files) will be transferred to the permanent data storage at the University of North Carolina at Greensboro (UNCG).
UNCG provides 25TB of Departmental Cloud Storage, where the data is institutionally owned and remains within the department's storage space even after the data creators have left the university.}

\textcolor{red}{Each publication or sub-project will have its own managed data package. These data packages will be versioned, with the initial submission package labeled as v.1.0. Any changes after the initial submission will be clearly indicated by updating the version code (e.g., major changes will update the first digit, while minor changes will update the second digit), along with a separate change log that details the modifications in each version.}

\textcolor{red}{PD Terui is responsible for overseeing the permanent data packages stored at UNCG.
Each PD is responsible for managing the source, intermediate, and final data copies related to their specific tasks:
\begin{itemize}
    \item University of Kansas (Hansen) -- data for Objective 1
    \item UNCG (Terui) -- data for Objectives 2 and 3
    \item University of Minnesota (Dolph, Finlay) -- original fish data
    \item The Nature Conservancy (Kennedy, Harlan, Piazza) -- products used for online tools
\end{itemize}}

\subsubsection*{Policy for Product Access and Sharing}

\textbf{Repositories}: All project products will be publicly accessible through either Zenodo or HydroShare.

Zenodo, a public repository that issues DOIs (Digital Object Identifiers) (https://zenodo.org/), provides a stable and citable location for data.
It also seamlessly integrates with GitHub, facilitating easy access to linked source code.
Additionally, Zenodo offers a version control system, enabling us to effectively manage any potential updates or modifications to published datasets as needed. Zenodo boasts a rich history of utilization in ecology and various other scientific domains.

HydroShare, maintained by the Consortium of Universities for the Advancement of Hydrologic Science, Inc. (CUAHSI), serves as a web-based hydrologic information system. CUAHSI, representing over 130 U.S. universities and international water-science-related organizations and sponsored by the National Science Foundation, supports the advancement of hydrologic science and education in the United States. HydroShare's mission is to provide a citable, shareable, and discoverable repository for the hydrology field's data and research products.

\textbf{Publication and Timing}: We will share our data and source codes following the publication of major findings from our project or, at the latest, within two years after the project's completion, whichever comes first.
This timeline ensures that our work remains accessible to the public and can benefit the research community promptly.

We acknowledge that most of the datasets we will utilize are publicly available. However, for any data that is state-owned and cannot be shared with the public, we will adhere to ethical and legal guidelines. We will provide references to the appropriate contacts, data archives, or publications that can guide readers on how to access or request the restricted data.

By adhering to these data management and sharing practices, our project aims to promote transparency, reproducibility, and open access to our computational resources and geospatial data. This approach ensures that our research can be leveraged by the broader scientific community and facilitates the validation and extension of our findings.

\end{document}