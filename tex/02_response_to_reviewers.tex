\documentclass[12pt, class=article, crop=false]{standalone}
\usepackage[subpreambles=true]{standalone}
\usepackage[T1]{fontenc} % for font setting
\usepackage{newtxtext,newtxmath}
\usepackage{import,
            graphicx,
            parskip,
            url,
            amsmath,
            wrapfig,
            fancyhdr}
            
\usepackage[most]{tcolorbox}
\usepackage[top=2.54cm, bottom=2.54cm, left=2.54cm, right=2.54cm, marginparwidth=2cm]{geometry}%set margin
\usepackage[sort&compress]{natbib}
\setcitestyle{square}
\setcitestyle{comma}
\usepackage[font={sf, small}, labelfont={bf, small}]{caption}

% header
\pagestyle{fancy}
\fancyhead[L]{Response to Previous Review}
\fancyhead[R]{}

\tcbuselibrary{breakable}
\bibliographystyle{bibstyle}
\graphicspath{{images/}}

\begin{document}
\section*{Response to Previous Review}

In light of the feedback provided last year, we have improved our proposal as follows:

\textbf{Complexities and Reliability of Publicly Available Data}:
The panel expressed concern that the input public data may introduce complexities and variations beyond the study's control, which could compromise the statistical ability to detect the true impact of conservation practices.
We agreed and revised the proposal as follows.

Our statistical ability to detect the impact may be influenced by (a) the data accuracy/consistency and (b) the model's statistical power.
For the data accuracy, we confirmed the reliability of the data sources.
Although our data sources are public, geospatial layers referenced in the proposal are peer-reviewed and quality-controlled, thus providing reliable input data for our statistical models (appropriate references added).
In addition, we confirmed that fish survey data were collected using a standard method across states, minimizing potential sampling biases.
One possible exception is the data source for stream chemistry (Water Quality Portal or WQP).
Multiple agencies populated WQP, potentially introducing data quality/consistency issues.
To address this potential problem, we introduced a quality-control process in the analysis of stream chemistry, in which we diagnose model residuals to detect suspicious data points.

The balance between the sample size and the model's complexity determines the model's statistical power.
In our project, this issue could be important in the Markov Network analysis, whose estimates will be used to construct energy landscapes.
In the revised proposal, we introduced a simulation approach to assess the appropriate sample size to detect the true effects of important predictors.
Based on the simulation results, we will assess the risk of Type II error (e.g., false rejection of conservation effects).
Previous simulation studies (Harris, 2016 Ecology 97: 3308--3314) have shown that the proposed model can detect true effects with moderate sample sizes ($\sim$ 50 samples), although the performance may vary by its complexity.
In our dataset, the number of sites in most watersheds exceeds this threshold; thus, we anticipate that this will have a minimal impact on the interpretation of our analysis.

\textbf{Data Storage}:
The panel expressed concern that our data storage may be insufficient to store the products from this project.
We revised our data management plan to clarify that (1) we have access to a 3 TB data storage in OneDrive, which is sufficient to store all the data sources temporarily (Terui confirmed); (2) we will use external hard drives as an additional back-up; and (3) we have access to a 25 TB permanent data storage (Departmental Cloud Storage), which will retain the project data permanently.
In addition, as indicated in the previous proposal, all research products will be archived in DOI-issuing public repositories, such as Zenodo. 

\textbf{Responsibilities of each PD in Data Management}:
The panel expressed concern that our data management plan lacks details about the roles and responsibilities of the PDs.
We clarified the role of each PD in Data Management Plan.
While each PD will keep file copies related to their tasks, the lead institution (University of North Carolina at Greensboro) will be responsible for securing all the source files and the derived products in the Departmental Cloud Storage.

\end{document}