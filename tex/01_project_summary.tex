\documentclass[12pt, class=article, crop=false]{standalone}
\usepackage[subpreambles=true]{standalone}
\usepackage[T1]{fontenc} % for font setting
\usepackage{tgtermes} % for font setting
\usepackage{import,
            graphicx,
            parskip,
            url,
            amsmath,
            wrapfig,
            fancyhdr}
\usepackage[most]{tcolorbox}
\usepackage[top=2.54cm, bottom=2.54cm, left=2.54cm, right=2.54cm, marginparwidth=2cm]{geometry}%set margin
\usepackage[sort&compress]{natbib}
\setcitestyle{square}
\setcitestyle{comma}
\usepackage[font={sf, small}, labelfont={bf, small}]{caption}

% header
\pagestyle{fancy}
\fancyhead[L]{Project Summary}
\fancyhead[R]{}

\tcbuselibrary{breakable}
\bibliographystyle{bibstyle}
\graphicspath{{images/}}

\begin{document}
\section*{Project Summary}

\textbf{Title:} Co-creating conservation landscapes to foster resilient aquatic communities in agricultural watersheds under a changing climate

Terui, Akira$^1$ (PD); Hansen, Amy T.$^2$ (Co-PD); Finlay, Jacques C.$^3$ (Co-PD); Dolph, Christine L.$^3$ (Co-PD); Piazza, Bryan$^4$ (Co-PD); Kennedy, Kathryn DM$^4$ (Co-PD); Harlan, David$^4$ (Co-PD)

$^1$ University of North Carolina at Greensboro;
$^2$ University of Kansas;
$^3$ University of Minnesota;
$^4$ The Nature Conservancy

Freshwater ecosystems provide ecosystem services essential to humans.
However, the provisioning of these ecosystem services is under threat due to anthropogenic climate change.
In particular, agricultural watersheds are highly vulnerable to climate change because environmental stressors (excess sediment and nutrients) interact with increased climate variability to affect freshwater organisms.
In the Midwestern US (Midwest), researchers have invested substantial efforts to mitigate the environmental impacts of agriculture through various approaches (cover crops, riparian buffer strips, and wetland restoration); these conservation practices have improved water quality by reducing nutrients and suspended sediment.
Yet, it remains unclear how these practices contribute to the ecological resilience of riverine biodiversity under climate change, leaving a critical knowledge gap that hinders the creation of sustainable agricultural landscapes in the Midwest.
The long-term goal of this project is to co-create resilient agricultural watersheds in the Midwest by synthesizing water resources research, biodiversity science, and proactive community engagement.
We formed a team with complementary expertise to achieve this goal with four supporting objectives: (1) Quantify
the impact of three major conservation practices (cover crops, riparian buffer strips, and wetlands) on surface water quality; (2) Develop a multispecies statistical framework linking conservation practices, water quality, and biodiversity; (3) Quantify the effects of land management and climate change on the resilience of riverine biodiversity across the Midwest; (4) Integrate research findings into existing stakeholder-driven decision support tools.

\end{document}

